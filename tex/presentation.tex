\documentclass[10pt]{beamer}

\usepackage{algorithm}
\usepackage[noend]{algorithmic}
\usepackage[english]{babel}
\usepackage{framed}
\usepackage[utf8]{inputenc}
\usepackage{minted}
\usepackage{xcolor}
\usepackage{xspace}

\usetheme{Copenhagen}
\setbeamertemplate{navigation symbols}{}
\setbeamertemplate{bibliography entry title}{}
\setbeamertemplate{bibliography entry location}{}
\setbeamertemplate{bibliography entry note}{}

% http://tex.stackexchange.com/a/151311/10806
\addtobeamertemplate{title page}{
  \includegraphics[scale=0.025]{tex/img/cern}
}{}

\title{
  Machine Learning Reference Extraction\\
  Using GROBID
}
\author[Jacopo Notarstefano]{
  Jacopo Notarstefano\\
  \texttt{jacopo.notarstefano [at] cern.ch}\\
  \vspace{0.25cm}
  \includegraphics[scale=0.25]{tex/img/inspire}
}
\date{October 14th, 2015}

% http://tex.stackexchange.com/a/4304/10806
\newcommand{\cpp}{C\nolinebreak\hspace{-.05em}\raisebox{.4ex}{\tiny\bf +}\nolinebreak\hspace{-.10em}\raisebox{.4ex}{\tiny\bf +}}

\definecolor{blue}{RGB}{51, 105, 232}
\definecolor{red}{RGB}{213, 15, 37}
\definecolor{yellow}{RGB}{238, 178, 17}
\definecolor{green}{RGB}{0, 153, 37}

\begin{document}
  \begin{frame}[plain]
    \titlepage
  \end{frame}

  \begin{frame}{Use cases}
    Curators often find themselves in one of the following situations:

    \vspace{0.25cm}

    \begin{enumerate}
      \item They only have the PDF of a paper, but no metadata.
      \item The source provides only part of the metadata, but they want to extract more from the PDF.
      \item They want to save the time spent entering metadata manually in BibEdit.
    \end{enumerate}

    \vspace{0.25cm}

    GROBID could help them solve these problems.
  \end{frame}

  \begin{frame}{GROBID}
    GROBID (GeneRation Of BIbliographical Data) is a machine learning library for parsing
    unstructured PDFs in structured XML documents, with a focus on technical and scientific
    publications.

    \vspace{0.5cm}

    GROBID is a Java library that wraps Wapiti, a \cpp\xspace toolkit for segmenting and
    labeling sequences.

    \vspace{0.5cm}

    GROBID is widely used, for example at ResearchGate, Mendeley, HAL...
  \end{frame}

  \begin{frame}{Labeling sequences and PDFs, 1/2}
    Let's see why extracting metadata from a PDF can be reduced to labeling a sequence.
    Let's consider for example a reference:

    \vspace{0.5cm}

    \begin{framed}
      G. Isidori and F. Teubert, \textit{Status of indirect searches for New Physics with heavy flavour decays after the initial LHC run, Eur.Phys.J.Plus} \textbf{129} (2014) 40
    \end{framed}
  \end{frame}

  \begin{frame}{Labeling sequences and PDFs, 2/2}
    Extracting metadata can be seen as labeling each word with a category,
    encoded in this case as a color:

    \vspace{0.5cm}

    \begin{framed}
      \color{blue}G. Isidori and F. Teubert, \color{red}\textit{Status of indirect searches for New Physics with heavy flavour decays after the initial LHC run, \color{green}Eur.Phys.J.Plus} \color{green}\textbf{129} (2014) 40\color{black}
    \end{framed}
  \end{frame}

  \begin{frame}{GROBID architecture}
    In GROBID parlance, the \emph{knowledge} that is used to extract metadata
    from data is called a \textbf{model}. GROBID is nothing more than a
    cascade of models, each acting on the output of the previous one.

    \vspace{0.5cm}

    \makebox[\textwidth][c]{
      \includegraphics[width=0.9\textwidth]{tex/img/cascade.pdf}
    }
  \end{frame}

  \begin{frame}{GROBID's output: TEI}
    TEI (Text Encoding Initiative) publishes a set of guidelines which specify
    encoding methods for machine-readable texts. By extension, we will call
    ``TEI'' the format described by these guidelines. GROBID's output conforms
    to a subset of TEI.

    \vspace{0.5cm}

    \inputminted[fontsize=\tiny]{xml}{tex/src/reference.xml}
  \end{frame}

  \begin{frame}{GROBID credits}
    GROBID is the work of Patrice Lopez, developer at INRIA. It has been adapted
    to papers from the HEP community by Joseph Boyd as part of his Master's Thesis
    for EPFL, under the supervision of Gilles Louppe, Senior Fellow at Inspire.

    \vspace{0.5cm}

    In particular, Joseph added to GROBID the concept of a \emph{collaboration},
    such as ATLAS or CMS, and improved considerably the accuracy of GROBID's
    models by providing lots of training data, taken from Inspire.
  \end{frame}

  \begin{frame}{Caveat emptor}
    \begin{framed}
      ``Converting PDF to XML is a bit like\\ converting hamburgers into cows.''

      \vspace{0.25cm}

      \hspace*\fill{--- Michael Kay}
    \end{framed}

    \vspace{0.5cm}

    That is, GROBID is not magic: it will misclassify things in various ways,
    and requires lots of training data to function properly.
  \end{frame}

  \begin{frame}{Invenio-Grobid}
    Unfortunately, the XML output is not compatible with what we use in Inspire.
    So we need to add some layer between GROBID and Invenio: \texttt{invenio-grobid}.

    \vspace{0.25cm}

    \texttt{invenio-grobid} is the joint work of me, my supervisor Jan Åge Lavik and
    Ilias Koutsakis.

    \vspace{0.25cm}

    \texttt{invenio-grobid} has been recently released on PyPI and is available at
    \url{https://github.com/inspirehep/invenio-grobid}.
  \end{frame}

  \begin{frame}[plain]
    \makebox[\textwidth][c]{
      \Huge DEMO
    }
  \end{frame}

  \begin{frame}{Current status and tentative roadmap, 1/2}
    \begin{enumerate}
      \setlength\itemsep{0.5cm}
      \item \begin{itemize}
              \item Allow catalogers to upload PDFs
              \item Automatic extraction of metadata
              \item Possibility to push to system
            \end{itemize}
      \item \begin{itemize}
              \item Allow catalogers to edit extracted metadata
              \item Perhaps integrate JSONEditor?
            \end{itemize}
    \end{enumerate}
  \end{frame}

  \begin{frame}{Current status and tentative roadmap, 2/2}
    \begin{enumerate}
      \setcounter{enumi}{2}
      \setlength\itemsep{0.25cm}
      \item \begin{itemize}
              \item Take over bibliographic reference parsing (i.e. kill Refextract)
	      \item Consolidate with other services (e.g. CrossRef)
	      \item Integrate in automatic ingestion workflows
            \end{itemize}
      \item \begin{itemize}
              \item Pre-fill user submission forms from PDF
            \end{itemize}
      \item \begin{itemize}
              \item \texttt{invenio-grobid} could be used to feed back corrections in GROBID, improving its precision
              \item Advanced visual interface to correct extraction (e.g. side-by-side view)
            \end{itemize}
    \end{enumerate}
  \end{frame}
\end{document}
